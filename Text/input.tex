
%\documentclass[a4paper,11pt]{article}

%einseitig
\documentclass[paper=a4, 11pt, cleardoubleempty, fancyheaders, oneside, openright, BCOR=7mm, DIV12, listof=totoc, index=totoc,bibliography=totoc, headinclude=false]{scrreprt}

%zweiseitig
%\documentclass[ngerman, paper=a4, 11pt,  twoside, openright, BCOR=7mm, DIV12, headinclude=false, listof=totoc, index=totoc,bibliography=totoc,headinclude=false,chapterprefix=on]{scrreprt}

\usepackage[english]{babel}

\usepackage{amsfonts}

\usepackage{amsmath}

\usepackage{amsthm}

\usepackage{amssymb}

\usepackage{paralist}

\usepackage{fancyhdr}

\usepackage{mathrsfs}

\usepackage{algorithm}

\usepackage[noend]{algpseudocode}

%\usepackage{hf-tikz}

%\usepackage[natbib=true,style=authoryear,backend=bibtex,useprefix=true]{biblatex}
\usepackage[natbib=true,style=IEEE,hyperref=true,backend=bibtex,useprefix=true]{biblatex}
%\usepackage{biblatex}
\addbibresource{sample.bib}

\usepackage{hyperref}

\pagestyle{fancy} %eigener Seitenstil
\renewcommand{\chaptermark}[1]{%
\markboth{\thechapter \ #1}{}}
\fancyhf{} %alle Kopf- und Fußzeilenfelder bereinigen
\fancyhead[C]{\leftmark} %zentrierte Kopfzeile
\renewcommand{\headrulewidth}{0.4pt} %obere Trennlinie
\renewcommand{\footrulewidth}{0.4pt} %untere Trennlinie
\fancyfoot[C]{\thepage} %Seitennummer

% Damit der Pagestyle auf bei Kapitelbeginn und bspw. bei TOC richtig gesetzt ist
\fancypagestyle{plain}{
\pagestyle{fancy} 
}


%\usepackage{showframe}           %zeigt Grenzen

%\numberwithin{equation}{section}

%\setlength{\topsep}{1ex} %aendern der Abstaende zwischen Sätzen/Propositionen/....
\theoremstyle{plain}% default
\newtheorem{thm}{Theorem}[chapter]
\newtheorem{lem}[thm]{Lemma}
\newtheorem{prop}[thm]{Proposition}
\newtheorem{cor}[thm]{Corollary}

\theoremstyle{definition}
\newtheorem{defn}[thm]{Definition}

\theoremstyle{remark}
\newtheorem{rema}[thm]{Bemerkung}