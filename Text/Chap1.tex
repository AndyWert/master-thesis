\chapter{Introduction}

When extracting oil from an oil reservoir, there are many parameters that influence the cost and the output. To achieve a greater profit, one wants to find parameters that optimize a net present value (NPV) function which evaluates the economic value of the chosen oil recovery strategy. A difficulty here is that the computation of this NPV function is usually expensive. Also, gradients are often not available. Therefore, one resorts to gradient free optimization methods. A popular example is the ensemble-based (EnOpt) optimization method which uses samples around each iterate to compute an approximation of the preconditioned gradient at that iterate. With this gradient, the next iterate is computed by a gradient ascent method.

In \cite{Keil2022-dj} a modification of the EnOpt method is proposed which is called the adaptive machine learning EnOpt (Adaptive-ML-EnOpt) algorithm. Here, a neural network-based surrogate is trained in each iteration to reduce costly calls of the full order model (FOM) NPV function and thus speed up the computation. This method is applied to an enhanced oil recovery strategy where a polymer-water mixture is injected into the reservoir to bring the oil to the surface. An extension of the EnOpt method is the so-called adaptive EnOpt algorithm. Here the covariance matrix that is used for the sampling around the iterates is adapted with respect to the result from the previous iteration.

In this thesis, we apply the adaptive EnOpt and the Adaptive-ML-EnOpt to an optimal control problem that is constrained by parabolic equations. The source code can be found on [github link]. In order to run this code, the python packages ...[packages with links] are necessary. The installation of CUDA[link] is optional but recommended if available.

The constrained control problem, along with the solution strategy of the corresponding discretized parabolic equation, is presented in chapter \ref{capterParabolicOptimalControlProblems}. After that, chapter \ref{ChapterEnsembleBasedOptimizationAlgorithm} presents the adaptive EnOpt algorithm. In chapter \ref{ChapterAdaptiveMLEnOptAlgorithm} the Adaptive-ML-EnOpt procedure from \cite{Keil2022-dj} is described. Finally, we test and compare these two algorithms in chapter \ref{chapterNumericalExperiments}.