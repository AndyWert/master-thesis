\chapter{\label{ChapterEnsembleBasedOptimizationAlgorithm}Ensemble‑based optimization algorithm}

The adaptive ensemble-based optimization (EnOpt) algorithm is usually used to maximize the net present value of oil recovery methods with respect to a control vector. Examples are presented in \cite{Keil2022-dj, OGUNTOLA2021109165, Zhang2023-sg}. In this chapter, we want to utilize the EnOpt algorithm to optimize the objective function $j$. Our implementation is similar to that in \cite{Keil2022-dj}.

We begin by describing this algorithm for a general function $F:\mathbb{R}^{N_\mathbf{q}}\to \mathbb{R}$ to iteratively solve the optimization problem
\begin{displaymath}
\operatorname*{maximize}_{\mathbf{q}\in\mathcal{D}}F(\mathbf{q}).
\end{displaymath}

We start at an initialization $\mathbf{q}_0$,  which is updated iteratively with a preconditioned gradient ascent method that is given by
\begin{displaymath}
\mathbf{q}_{k+1} = \mathbf{q}_k+\beta_k\mathbf{d}_k,
\end{displaymath}
\begin{displaymath}
\mathbf{d}_k\approx\frac{\mathbf{C}_{\mathbf{q}_k}^k\mathbf{G}_k}{\|\mathbf{C}_{\mathbf{q}_k}^k\mathbf{G}_k\|}_\infty,
\end{displaymath}
where $k=0,1,2,\dotsc$ denotes the optimization iteration. $\beta_k$ with $\beta_k>0$ is computed by using a line search. Furthermore, $\mathbf{C}_{\mathbf{q}_k}^k$ denotes the user-defined covariance matrix of the control variables at the $k$-th iteration and $\mathbf{G}_k$ is the approximate gradient of $F$ with respect to the control variables.

We define the initial covariance matrix $\mathbf{C}_{\mathbf{q}_0}^0$ so that the covariance between controls of different basis functions $\phi_i,\phi_j$ is zero and
\begin{equation}
\label{defineInitialCovariance}
\mathrm{Cov}(q_j^i,q_j^{i+h})=\sigma_j^2\rho^h\left(\frac{1}{1-\rho^2}\right),\text{ for all }h\in\{0,\dotsc,M-i\},
\end{equation}
where $\sigma_j^2>0$ is the variance for the basis function $\phi_j$ and $\rho\in(-1,1)$ the correlation coefficient.

That means that for $\mathbf{C}_j:=\left(\mathrm{Cov}(q_j^i,q_j^{k})\right)_{i,k}$ with $j=1,\dotsc,N_b$, we set
\begin{equation}
\label{initCov}
\mathbf{C}_{\mathbf{q}_0}^0 =
\begin{pmatrix}
  \mathbf{C}_1 & 0 & \cdots & 0 \\
  0 & \mathbf{C}_2 & \cdots & 0 \\
  \vdots  & \vdots  & \ddots & \vdots  \\
  0 & 0 & \cdots & \mathbf{C}_{N_b} 
 \end{pmatrix}.
\end{equation}

To compute the step direction $\mathbf{d}_k$ at iteration step $k$, we sample $\mathbf{q}_{k,m}\in\mathcal{D}$ for $m=1,\dotsc,N$, with $N\in\mathbb{N}$, from a multivariate Gaussian distribution with the mean $\mathbf{q}_k$ and the covariance $\mathbf{C}_{\mathbf{q}_k}^k$. Then we define
\begin{equation}
\label{CqkF}
\mathbf{C}_{\mathbf{q}_k,F}^k:=\frac{1}{N-1}\sum_{m=1}^N(\mathbf{q}_{k,m}-\mathbf{q}_k)(F(\mathbf{q}_{k,m})-F(\mathbf{q}_k)).
\end{equation}
Now we set $\mathbf{d}_k=\frac{\mathbf{C}_{\mathbf{q}_k,F}^k}{\|\mathbf{C}_{\mathbf{q}_k,F}^k\|_\infty}$. This is valid since $\mathbf{C}_{\mathbf{q}_k,F}^k$ is an estimation of $\mathbf{C}_{\mathbf{q}_k}^k\mathbf{G}_k$, which can by shown like in \cite{OGUNTOLA2021109165}. Here, we begin with the Taylor expansion around $\mathbf{q}_k$ and get
\begin{eqnarray*}
&&F(\mathbf{q})=F(\mathbf{q}_k)+(\mathbf{q}-\mathbf{q}_k)^T\nabla F(\mathbf{q}_k)+O(\|\mathbf{q}-\mathbf{q}_k\|^2)\\
&\implies&F(\mathbf{q})-F(\mathbf{q}_k)=(\mathbf{q}-\mathbf{q}_k)^T\mathbf{G}_k+O(\|\mathbf{q}-\mathbf{q}_k\|^2).
\end{eqnarray*}
Multiplying both sides by $(\mathbf{q}-\mathbf{q}_k)$ and setting $\mathbf{q}=\mathbf{q}_{k,m}$ yields
\begin{eqnarray*}
&&(\mathbf{q}_{k,m}-\mathbf{q}_k)(F(\mathbf{q}_{k,m})-F(\mathbf{q}_k))\\
&=&(\mathbf{q}_{k,m}-\mathbf{q}_k)(\mathbf{q}_{k,m}-\mathbf{q}_k)^T\mathbf{G}_k+O(\|\mathbf{q}_{k,m}-\mathbf{q}_k\|^3),
\end{eqnarray*}
where $O(\|\mathbf{q}_{k,m}-\mathbf{q}_k\|^3)$ are the remaining terms containing order $\geq3$ of $(\mathbf{q}_{k,m}-\mathbf{q}_k)$. Neglecting $O(\|\mathbf{q}_{k,m}-\mathbf{q}_k\|^3)$ gives by summation over all samples and multiplication of both sides with $\frac{1}{N-1}$:
\begin{eqnarray*}
&&\frac{1}{N-1}\sum_{m=1}^N(\mathbf{q}_{k,m}-\mathbf{q}_k)(F(\mathbf{q}_{k,m})-F(\mathbf{q}_k))\\
&\approx&\left(\frac{1}{N-1}\sum_{m=1}^N(\mathbf{q}_{k,m}-\mathbf{q}_k)(\mathbf{q}_{k,m}-\mathbf{q}_k)^T\right)\mathbf{G}_k\\
&\implies&\mathbf{C}_{\mathbf{q}_k,F}^k\approx\mathbf{C}_{\mathbf{q}_k}^k\mathbf{G}_k,
\end{eqnarray*}
since $\frac{1}{N-1}\sum_{m=1}^N(\mathbf{q}_{k,m}-\mathbf{q}_k)(\mathbf{q}_{k,m}-\mathbf{q}_k)^T$ is itself an approximation of $\mathbf{C}_{\mathbf{q}_k}^k$.\\

By using the samples $\{\mathbf{q}_{k-1,m}\}_{m=1}^N$ and the covariance matrix $\mathbf{C}_{\mathbf{q}_{k-1}}^{k-1}$ from the last iteration, we update $\mathbf{C}_{\mathbf{q}_{k-1}}^{k-1}$, like in \cite{Stordal2016-cj}, by setting
\begin{equation}
\label{updateCov}
\mathbf{C}_{\mathbf{q}_{k}}^{k}=\mathbf{C}_{\mathbf{q}_{k-1}}^{k-1}+\tilde{\beta}_k\tilde{\mathbf{d}}_k\text{ with}
\end{equation}
\begin{equation}
\label{updateCovDirection}
\tilde{\mathbf{d}}_k=N^{-1}\sum_{m=1}^N(F(\mathbf{q}_{k-1,m})-F(\mathbf{q}_k))((\mathbf{q}_{k-1,m}-\mathbf{q}_k)(\mathbf{q}_{k-1,m}-\mathbf{q}_k)^T-\mathbf{C}_{\mathbf{q}_{k-1}}^{k-1}),
\end{equation}
where $\tilde{\beta}_k$ is a step size that is chosen so that no entries of the diagonal of $\mathbf{C}_{\mathbf{q}_{k}}^{k}$ are negative. How we set $\tilde{\beta}_k$ is shown below at the implementation of the entire EnOpt algorithm.

Now that we have described the optimization steps of this algorithm, we iterate until $F(\mathbf{q}_k)\leq F(\mathbf{q}_{k-1})+\varepsilon$, where $\varepsilon>0$.\\

In our implementation, the EnOpt algorithm takes the objective function $\textproc{F}:\mathbb{R}^{N_\mathbf{q}}\to\mathbb{R}$, our initial iterate $\mathbf{q}_0\in\mathbb{R}^{N_\mathbf{q}}$, the sample size $N\in\mathbb{N}$, the tolerance $\varepsilon>0$, the maximum number of iterations $k^*\in\mathbb{N}$, the initial step size $\beta_1>0$ for the computation of the next iterate, the initial step size $\beta_2>0$ for the iteration of the covariance matrix, the step size contraction $r\in(0,1)$, the maximum number of step size trials $\nu^*\in\mathbb{N}$, the variance $\sigma^2\in\mathbb{R}^{N_b}$ with positive elements, the correlation coefficient $\rho\in(-1,1)$, the number of time steps $N_t\in\mathbb{N}$, the number of basis functions $N_b\in\mathbb{N}$, a projection $\textproc{pr}$ and an initial covariance $\mathbf{C}_\mathrm{init}\in\mathbb{R}^{N_\mathbf{q}}$. $\textproc{pr}$ is set to the identity function $\mathbf{lambda}\:\mathrm{mu}: \mathrm{mu}$ and $\mathbf{C}_\mathrm{init}$ is set to $\mathrm{None}$ by default.

$\textproc{pr}$ is used to project the inputs onto a given set and $\mathbf{C}_\mathrm{init}$ is an alternative definition for the initialization of the covariance. In this chapter we do not need to specify these inputs, however $\textproc{pr}$ could be used if we had an optimization problem where the iterates were restricted to a certain spatial domain, which is here not the case.

The implementation in pseudo code is shown below:

%\begin{algorithm}[H]%\footnotesize
%\caption{EnOpt algorithm}
%\begin{algorithmic}[1]
%\Function{EnOpt}{$F,\mathbf{q}_0,N,\varepsilon,k^*,\beta,\tilde{\beta},r,\nu^*,\sigma^2,\rho,\operatorname{pr}=\operatorname{id}$}
%\State $F_{{0}}\gets F(\mathbf{q}_0)$
%\State $\mathbf{q}_{1},T_{1},\mathbf{C}_{\mathbf{q}_{0}}^{0},F_{1}\gets\mathrm{OptStep}(F,\mathbf{q}_0,N,0,\{\},0,F_{0},\beta,\tilde{\beta},r,\varepsilon,\nu^*,\sigma^2,\rho,\operatorname{pr})$
%\State $k\gets 1$
%\While{$F_{k}>F_{k-1}+\varepsilon$\text{ and }$k<k^*$}
%\State $\mathbf{q}_{k+1},T_{k+1},\mathbf{C}_{\mathbf{q}_{k}}^{k},F_{k+1}\gets\mathrm{OptStep}(F,\mathbf{q}_k,N,k,T_k,\mathbf{C}_{\mathbf{q}_{k-1}}^{k-1},F_k,\beta,\tilde{\beta},r,\varepsilon,\nu^*,\sigma^2,\rho,\operatorname{pr})$
%\State $k\gets k+1$
%\EndWhile
%\State \Return $\mathbf{q}^*\gets\mathbf{q}_k$
%\EndFunction
%\end{algorithmic}
%\end{algorithm}
\begin{algorithm}[H]%\footnotesize
\caption{\label{EnOptAlg}EnOpt algorithm}
\begin{algorithmic}[1]
\Function{EnOpt}{$\textproc{F},\mathbf{q}_0,N,\varepsilon,k^*,\beta_1,\beta_2,r,\nu^*,\sigma^2,\rho, N_t, N_b, \textproc{pr} \gets \mathbf{lambda}\:\mathrm{mu}: \mathrm{mu}, \mathbf{C}_\mathrm{init}\gets\mathrm{None}$}
\State $F^\mathrm{prev}_{k}\gets \Call{F}{\mathbf{q}_0}$
\State\label{EnOptAlgOptStepCall1} $\mathbf{q}_k,T_k,\mathbf{C}_k,F_k\gets\Call{OptStep}{\textproc{F},\mathbf{q}_0,N,0,[\;],\mathbf{C}_\mathrm{init},F^\mathrm{prev}_{k},\beta_1,\beta_2,r,\varepsilon,\nu^*,\sigma^2,\rho, N_t, N_b, \textproc{pr}}$
\State $k\gets 1$
\While{$F_{k}>F^\mathrm{prev}_k+\varepsilon$\text{ and }$k<k^*$}
\State $F^\mathrm{prev}_k\gets F_k$
\State\label{EnOptAlgOptStepCall2} $\mathbf{q}_{k},T_{k},\mathbf{C}_k,F_{k}\gets\Call{OptStep}{\textproc{F},\mathbf{q}_k,N,k,T_k,\mathbf{C}_k,F_k,\beta_1,\beta_2,r,\varepsilon,\nu^*,\sigma^2,\rho, N_t, N_b,\textproc{pr}}$
\State $k\gets k+1$
\EndWhile
\State \Return $\mathbf{q}_k, k$
\EndFunction
\end{algorithmic}
\end{algorithm}

We begin by initializing $F^\mathrm{prev}_{k}$ as $\Call{F}{\mathbf{q}_0}$. The next iterate $\mathbf{q}_k$, along with the sample $T_k$, the covariance $\mathbf{C}_k$ and the function value of $\mathbf{q}_k$, denoted by $F_k$, are computed by calling the initial optimization step $\textproc{OptStep}$, which is shown in algorithm \ref{OptStep}.

After we initialized $k$ as $1$, we loop until the stop criterion is satisfied. In each optimization loop, $F^\mathrm{prev}_{k}$ is set to the function value of the last iterate and $\mathbf{q}_k,T_k,\mathbf{C}_k$ and $F_k$ are updated by calling $\textproc{OptStep}$ again.

The next algorithms use some functions from Python such as $\textproc{len}$ for the length of a list or an array, as well as functions from the Python package NumPy, which are identified by starting with '$\mathrm{np.}$'.

%\begin{algorithm}[H]%\footnotesize
%\caption{OptStep algorithm}
%\begin{algorithmic}[1]
%\Function{OptStep}{$F,\mathbf{q}_k,N,k,T_k,\mathbf{C}_{\mathbf{q}_{k-1}}^{k-1},F_k,\beta,\tilde{\beta},r,\varepsilon,\nu^*,\sigma^2,\rho,\operatorname{pr}$}
%\If{$k=0$}
%\State \text{Compute the initial covariance matrix }$\mathbf{C}_{\mathbf{q}_0}^0$\text{ like defined in }\eqref{initCov}\text{ with }$\mathbf{q}_0,\sigma^2,\rho$
%\Else
%\State $\mathbf{C}_{\mathbf{q}_{k}}^{k} \gets \mathrm{updateCov}(\mathbf{q}_k, N, T_k, \mathbf{C}_{\mathbf{q}_{k-1}}^{k-1}, F_k ,\tilde{\beta})$
%\EndIf
%\State \text{Sample }$N$\text{ control vectors }$\{\mathbf{q}_{k,j}\}_{j=1}^N$\text{ from a distribution }$\mathcal{N}(\mathbf{q}_{k},\mathbf{C}_{\mathbf{q}_{k}}^{k})$
%\State $T_{k+1}\gets\{\mathbf{q}_{k,j},F(\mathbf{q}_{k,j})\}_{j=1}^N$
%\State \text{Compute the vector }$\mathbf{C}_{\mathbf{q}_{k},F}^{k}$\text{ with }$\mathbf{q}_k,F_k$\text{ and the stored values of }$T_{k+1}\text{ like in }\eqref{CqkF}$
%\State \text{Compute the search direction }$\mathbf{d}_k=\mathbf{C}_{\mathbf{q}_{k},F}^{k}/\|\mathbf{C}_{\mathbf{q}_{k},F}^{k}\|_\infty$
%\State $\mathbf{q}_{k+1}\gets\mathrm{LineSearch}(F,\mathbf{q}_k,\mathbf{d}_k,\beta,r,\varepsilon,\nu^*,\operatorname{pr})$
%\State $F_{k+1}\gets F(\mathbf{q}_{k+1})$
%\State \Return $\mathbf{q}_{k+1},T_{k+1},\mathbf{C}_{\mathbf{q}_k}^k,F_{k+1}$
%\EndFunction
%\end{algorithmic}
%\end{algorithm}
\begin{algorithm}[H]%\footnotesize
\caption{\label{OptStep}OptStep algorithm}
\begin{algorithmic}[1]
\Function{OptStep}{$\textproc{F},\mathbf{q}_k,N,k,T_k,\mathbf{C}_k,F_k,\beta_1,\beta_2,r,\varepsilon,\nu^*,\sigma^2,\rho, N_t, N_b, \textproc{pr} \gets \mathbf{lambda}\:\mathrm{mu}: \mathrm{mu}$}
\State $N_\mathbf{q}\gets\Call{len}{\mathbf{q}_k}$
\State $\mathbf{C}^\mathrm{next}_k \gets \mathrm{np.}\Call{zeros}{(N_\mathbf{q}, N_\mathbf{q})}$
\If{$k=0$}
\If{$\mathbf{C}_k$\text{ is }$\mathrm{None}$}
\State $\mathbf{C}^\mathrm{next}_k \gets \Call{initCov}{\protect\Call{len}{\mathbf{q}_k}, \sigma^2, \rho, N_t, N_b}$
\Else
\State $\mathbf{C}^\mathrm{next}_k \gets \mathbf{C}_k.\Call{copy}{\:}$
\EndIf
\Else
\State $\mathbf{C}^\mathrm{next}_k \gets \Call{updateCov}{\mathbf{q}_k, T_k, \mathbf{C}_k, F_k ,\beta_2}$
\EndIf
\State $\mathrm{sample} \gets \mathrm{np.random.}\Call{multivariate\_normal}{\mathbf{q}_k, \mathbf{C}^\mathrm{next}_k, \mathrm{size} \gets N}$
\State $T^\mathrm{next}_k\gets[\Call{pr}{\mathrm{sample}[j]},\Call{F}{\protect\Call{pr}{\mathrm{sample}[j]}}]_{j=0}^{N-1}$
\State $\mathbf{C}^k_F \gets \mathrm{np.}\Call{zeros}{N_\mathbf{q}}$
\For{$m=0,\dotsc,N-1$}
\State $\mathbf{C}^k_F \gets \mathbf{C}^k_F+(T^\mathrm{next}_k[m][0]-\mathbf{q}_k)\cdot(T^\mathrm{next}_k[m][1]-F_k)$
\EndFor
\State $\mathbf{C}^k_F \gets 1/(N-1)\cdot\mathbf{C}^k_F$
\State $\mathbf{d}_k \gets \mathrm{np.}\Call{zeros}{N_\mathbf{q}}$
\State $\mathbf{q}^\mathrm{next}_{k}, F^\mathrm{next}_k \gets \mathrm{np.}\Call{copy}{\mathbf{q}_k},F_k$
\If{{\bfseries not }$\mathrm{np.}\Call{all}{\mathbf{C}^k_F==0}$}
\State $\mathbf{d}_k \gets \mathbf{C}^k_F/\|\mathbf{C}^k_F\|_\infty$
\State $\mathbf{q}^\mathrm{next}_{k}, F^\mathrm{next}_k \gets \Call{LineSearch}{\textproc{F},\mathbf{q}_k,F_k,\mathbf{d}_k,\beta_1,r,\varepsilon,\nu^*,\textproc{pr}}$
\EndIf
\State \Return $\mathbf{q}^\mathrm{next}_k,T^\mathrm{next}_k,\mathbf{C}^\mathrm{next}_k,F^\mathrm{next}_k$
\EndFunction
\end{algorithmic}
\end{algorithm}

We start the $\textproc{OptStep}$ algorithm by updating the covariance matrix $\mathbf{C}_k$ or, if $k$ is zero, initializing it. In the case that $k$ is zero, we first check if there is a predefined covariance matrix $\mathbf{C}_k$. When there is one, $\mathbf{C}^\mathrm{next}_k$ is set to that matrix. Otherwise, we define $\mathbf{C}^\mathrm{next}_k$ by calling the function $\textproc{initCov}$, which sets the matrix like it is described in \eqref{initCov}.

If $k$ is not zero, we get the updated covariance matrix by calling $\textproc{updateCov}$, which is described in \eqref{updateCov} and algorithm \ref{updateCovAlg}.

This covariance matrix is now used to get a Gaussian distributed sample with $N$ elements around $\mathbf{q}_k$. The projected samples $\Call{pr}{\mathrm{sample}[j]}$ and their respective function values $\Call{F}{\protect\Call{pr}{\mathrm{sample}[j]}}$ for $j=0,\dotsc,N-1$ are stored in the list $T^\mathrm{next}_k$. The definition is here abbreviated as $[\Call{pr}{\mathrm{sample}[j]},\Call{F}{\protect\Call{pr}{\mathrm{sample}[j]}}]_{j=0}^{N-1}$. In the code, this is done with an iteration through a for-loop.

After that, we set $\mathbf{C}^k_F$ like it is defined in \eqref{CqkF}. Then we check if $\mathbf{C}^k_F$ is equal to the zero vector. That can happen if the functional F is constant on the sample set. In that case, we let the next iterate $\mathbf{q}^\mathrm{next}_k$ and the corresponding functional value $F_k$ unchanged.

Otherwise, we set $\mathbf{d}_k$ to $\mathbf{C}^k_F$ divided by its sup norm. Here, $\|\mathbf{C}^k_F\|_\infty$ is an abbreviation of $\mathrm{np.}\Call{max}{\mathrm{np.}\protect\Call{abs}{\mathbf{C}^k_F}}$. The next iterate $\mathbf{q}^\mathrm{next}_{k}$ and its functional value $F^\mathrm{next}_k$ is now computed with the line search algorithm $\textproc{LineSearch}$, that is shown in algorithm \ref{LineSearchAlg}.

We return $\mathbf{q}^\mathrm{next}_k,T^\mathrm{next}_k,\mathbf{C}^\mathrm{next}_k$ and $F^\mathrm{next}_k$.

%\begin{algorithm}[H]%\footnotesize
%\caption{Covariance matrix initialization}
%\begin{algorithmic}[1]
%\Function{initCov}{$\mathbf{q}_k, \sigma^2, \rho$}
%\State $N_\mathbf{q}\gets \mathrm{len}(\mathbf{q}_k)$
%\State \Return $$
%\EndFunction
%\end{algorithmic}
%\end{algorithm}

%\begin{algorithm}[H]%\footnotesize
%\caption{Covariance matrix update}
%\begin{algorithmic}[1]
%\Function{updateCov}{$\mathbf{q}_k, N, T_k, \mathbf{C}_{\mathbf{q}_{k-1}}^{k-1}, F_k ,\tilde{\beta}$}
%\State $N_\mathbf{q}\gets \mathrm{len}(\mathbf{q}_k)$
%\State $\mathbf{d}_k \gets \mathrm{np.zeros}((N_\mathbf{q}, N_\mathbf{q}))$
%\State $\mathrm{assert}$ $\mathrm{len}(T_k) == N$
%\For{$m=0,\dotsc,N-1$}
%\State $\mathbf{d}_k \gets d_k + (T_k[m][1]-F_k)*((T_k[m][0]-\mathbf{q}_k).\mathrm{reshape}((N_\mathbf{q},1))*(T_k[m][0]-\mathbf{q}_k).\mathrm{reshape}((1,N_\mathbf{q}))-\mathbf{C}_{\mathbf{q}_{k-1}}^{k-1})$
%\EndFor
%\State $\mathbf{d}_k \gets \mathbf{d}_k/N$
%\State $\mathbf{C}_\mathrm{diag}\gets\mathrm{np.zeros}(N_\mathbf{q})$
%\State $\mathbf{d}_\mathrm{diag}\gets\mathrm{np.zeros}(N_\mathbf{q})$
%\For{$i=0,\dotsc,N_\mathbf{q}-1$}
%\State $\mathbf{C}_\mathrm{diag}[i]\gets\mathbf{C}_{\mathbf{q}_{k-1}}^{k-1}[i, i]$
%\State $\mathbf{d}_\mathrm{diag}[i]\gets\mathbf{d}_k[i, i]$
%\EndFor
%\State $\tilde{\beta}_k\gets\tilde{\beta}$
%\While{$\mathrm{np.min}(\mathbf{C}_\mathrm{diag}+\tilde{\beta}_k*\mathbf{d}_k)<0$}
%\State $\tilde{\beta}_k \gets \tilde{\beta}_k/2$
%\EndWhile
%\State \Return $\mathbf{C}_{\mathbf{q}_{k-1}}^{k-1}+\tilde{\beta}_k*\mathbf{d}_k$
%\EndFunction
%\end{algorithmic}
%\end{algorithm}
\begin{algorithm}[H]%\footnotesize
\caption{\label{updateCovAlg}Covariance matrix update}
\begin{algorithmic}[1]
\Function{updateCov}{$\mathbf{q}_k, T_k, \mathbf{C}_k, F_k ,\beta_2$}
\State $N_\mathbf{q}\gets \Call{len}{\mathbf{q}_k}$
\State $N\gets \Call{len}{T_k}$
\State $\mathbf{d}^\mathrm{cov}_k \gets \Call{np.zeros}{(N_\mathbf{q}, N_\mathbf{q})}$
%\State $\mathrm{assert}$ $\mathrm{len}(T_k) == N$
\For{$m=0,\dotsc,N-1$}
\State $\mathbf{d}^\mathrm{cov}_k \gets \mathbf{d}^\mathrm{cov}_k + (T_k[m][1]-F_k)\cdot((T_k[m][0]-\mathbf{q}_k).\Call{reshape}{(N_\mathbf{q},1)}\cdot(T_k[m][0]-\mathbf{q}_k).\Call{reshape}{(1,N_\mathbf{q})}-\mathbf{C}_k)$
\EndFor
\State $\mathbf{d}^\mathrm{cov}_k \gets \mathbf{d}^\mathrm{cov}_k/N$
\State $\mathbf{C}_\mathrm{diag}, \mathbf{d}_\mathrm{diag}\gets\mathrm{np.}\Call{zeros}{N_\mathbf{q}}, \mathrm{np.}\Call{zeros}{N_\mathbf{q}}$
\For{$i=0,\dotsc,N_\mathbf{q}-1$}
\State $\mathbf{C}_\mathrm{diag}[i]\gets\mathbf{C}_k[i, i]$
\State $\mathbf{d}_\mathrm{diag}[i]\gets\mathbf{d}^\mathrm{cov}_k[i, i]$
\EndFor
\State $\beta^\mathrm{iter}_2\gets\beta_2$
\While{$\mathrm{np.}\Call{min}{\mathbf{C}_\mathrm{diag}+\beta^\mathrm{iter}_2\cdot\mathbf{d}_\mathrm{diag}}\leq0$}
\State $\beta^\mathrm{iter}_2 \gets \beta^\mathrm{iter}_2/2$
\EndWhile
\State \Return $\mathbf{C}_k+\beta^\mathrm{iter}_2*\mathbf{d}^\mathrm{cov}_k$
\EndFunction
\end{algorithmic}
\end{algorithm}

For the update of the covariance matrix, we calculate the step direction $\mathbf{d}^\mathrm{cov}_k$ like in \ref{updateCovDirection} first. Then we want to make sure that the updated covariance matrix has only positive values on its diagonal.

For this purpose, $\mathbf{C}_\mathrm{diag}$ and $\mathbf{d}_\mathrm{diag}$ are defined as the vector of values on the diagonal of $\mathbf{C}_k$ and $\mathbf{d}^\mathrm{cov}_k$ respectively. Then, the step size $\beta^\mathrm{iter}_2$, initialized as $\beta_2$, is halfed until $\mathrm{np.}\Call{min}{\mathbf{C}_\mathrm{diag}+\beta^\mathrm{iter}_2*\mathbf{d}_\mathrm{diag}}$ is positive. Finally, the matrix $\mathbf{C}_k+\beta^\mathrm{iter}_2*\mathbf{d}^\mathrm{cov}_k$ is returned, like in \eqref{updateCov}.

%\begin{algorithm}[H]%\footnotesize
%\caption{Line search}
%\begin{algorithmic}[1]
%\Function{LineSearch}{$F,\mathbf{q}_k,\mathbf{d}_k,\beta,r,\varepsilon,\nu^*,\operatorname{pr}$}
%\State $\beta_k \gets \beta$
%\State $\mathbf{q}_{k+1} \gets \operatorname{pr}(\mathbf{q}_k+\beta_k\mathbf{d}_k)$
%\State $\nu \gets 0$
%\While{$F(\mathbf{q}_{k+1})-F(\mathbf{q}_k)\leq\varepsilon$\text{ and }$\nu<\nu^*$}
%\State $\beta_k \gets r\beta_k$
%\State $\mathbf{q}_{k+1} \gets \operatorname{pr}(\mathbf{q}_k+\beta_k\mathbf{d}_k)$
%\State $\nu \gets \nu+1$
%\EndWhile
%\Return $\mathbf{q}_{k+1}$
%\EndFunction
%\end{algorithmic}
%\end{algorithm}
\begin{algorithm}[H]%\footnotesize
\caption{\label{LineSearchAlg}Line search}
\begin{algorithmic}[1]
\Function{LineSearch}{$\textproc{F}, \mathbf{q}_k, F_k, \mathbf{d}_k, \beta_1, r, \varepsilon, \nu^*, \textproc{pr}$}
\State $\beta^\mathrm{iter}_1 \gets \beta_1$
\State $\mathbf{q}^\mathrm{next}_k \gets \Call{pr}{\mathbf{q}_k+\beta^\mathrm{iter}_1\mathbf{d}_k}$
\State $F^\mathrm{next}_k \gets \Call{F}{\mathbf{q}^\mathrm{next}_k}$
\State $\nu \gets 0$
\While{$F^\mathrm{next}_k-F_k\leq\varepsilon$\text{ and }$\nu<\nu^*$}
\State $\beta^\mathrm{iter}_1 \gets r\beta^\mathrm{iter}_1$
\State $\mathbf{q}^\mathrm{next}_k \gets \Call{pr}{\mathbf{q}_k+\beta^\mathrm{iter}_1\mathbf{d}_k}$
\State $F^\mathrm{next}_k \gets \Call{F}{\mathbf{q}^\mathrm{next}_k}$
\State $\nu \gets \nu+1$
\EndWhile
\Return $\mathbf{q}^\mathrm{next}_k, F^\mathrm{next}_k$
\EndFunction
\end{algorithmic}
\end{algorithm}

At the start of the line search algorithm, we initialize the step size $\beta^\mathrm{iter}_1$ as $\beta_1$. Then we repeatedly calculate $\mathbf{q}^\mathrm{next}_k = \Call{pr}{\mathbf{q}_k+\beta^\mathrm{iter}_1\mathbf{d}_k}$ and $F^\mathrm{next}_k = \Call{F}{\mathbf{q}^\mathrm{next}_k}$ with simultaneous reduction of $\beta^\mathrm{iter}_1$ by multiplication with $r$ until either $F^\mathrm{next}_k-F_k>\varepsilon$ or $\nu\geq\nu^*$.

After the termination of the while-loop, $\mathbf{q}^\mathrm{next}_k$ and $F^\mathrm{next}_k$ are returned. The reason for stopping the while loop also shows when $\mathbf{q}^\mathrm{next}_k$ is the last iteration of the EnOpt algorithm, as the EnOpt algorithm stops when the function value of the next iteration is not greater than the function value of the last iteration plus $\varepsilon$, i.e. when $F^\mathrm{next}_k>F_k+\varepsilon$.

Now we use this algorithm to optimize our objective function $j$. Since this is a maximization procedure and $j$ should be minimized, we apply $-j$, denoted by $-\textproc{j}$, to the EnOpt algorithm, which gives us:

%\begin{algorithm}[H]%\footnotesize
%\caption{FOM-EnOpt algorithm}
%\begin{algorithmic}[1]
%\Function{FOM-EnOpt}{$\mathbf{q}_0,N,\varepsilon,k^*,\beta,\tilde{\beta},r,\nu^*,\sigma^2,\rho$}
%\State \Return $\mathrm{EnOpt}(-j,\mathbf{q}_0,N,\varepsilon,k^*,\beta,\tilde{\beta},r,\nu^*,\sigma^2,\rho)$
%\EndFunction
%\end{algorithmic}
%\end{algorithm}
\begin{algorithm}[H]%\footnotesize
\caption{\label{FOM-EnOpt}FOM-EnOpt algorithm}
\begin{algorithmic}[1]
\Function{FOM-EnOpt}{$\mathbf{q}_0,N,\varepsilon,k^*,\beta_1,\beta_2,r,\nu^*,\sigma^2,\rho, N_t, \mathbf{q}_\mathrm{base}$}
\State \Return $\Call{EnOpt}{-\textproc{j},\mathbf{q}_0,N,\varepsilon,k^*,\beta_1,\beta_2,r,\nu^*,\sigma^2,\rho, N_t, \protect\Call{len}{\mathbf{q}_\mathrm{base}}}$
\EndFunction
\end{algorithmic}
\end{algorithm}

$\mathbf{q}_\mathrm{base}$ is here a list that consists of the basis functions, so $\Phi$ in \eqref{basisFuncionsList}. There are some more inputs that $\textproc{FOM-EnOpt}$ requires, but we omit these as they are only needed for the calculation of $\textproc{j}$.